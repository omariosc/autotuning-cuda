\documentclass[a4paper, draft]{article}

% Common header definitions and settings for use in the other documentation.


% Packages
\usepackage[T1]{fontenc}    % Font Encoding
\usepackage{utopia}         % Font
%\usepackage{courier}        % Monospaced font % This messes up the code listings, as it is wider than the default.
\usepackage{titling}        % Make modifications to \maketitle
\usepackage{fancybox}       % Boxes around terminal listings.
\usepackage{fancyvrb}       % Code listings, terminal sessions, etc.
\usepackage{color}          % Define and use colours
\usepackage{tocloft}        % TOC modification
\usepackage{datetime}       % Date formatting
\usepackage{float}          % Define new float types (code)
\usepackage{caption}        % Modify caption appearance
\usepackage{todonotes}      % Add todo notes in the margin
\usepackage{framed}         % shaded environment provides background colours
\usepackage{fourier-orns}   % Ornaments around title
\usepackage{siunitx}        % Align on decimal point column type
\usepackage{ifdraft}        % Test for draft mode
\usepackage{tikz}           % Drawing Trees
\usepackage{pdflscape}      % provides landscape environment for pdflatex.
\usepackage{ifthen}         % Provides conditional tests
\usepackage{hyperref}       % Links, and nameref command




% Define some commannds for displaying the names of different things
\newcommand{\filename}[1]{\texttt{#1}}
\newcommand{\command}[1]{\texttt{#1}}
\newcommand{\var}[1]{$#1$}
\newcommand{\mathvar}[1]{#1}
\newcommand{\codefragment}[1]{\texttt{#1}}
\newcommand{\confsnippet}[1]{\texttt{#1}}

\newcommand{\class}[1]{\texttt{#1}}
\newcommand{\func}[1]{\texttt{#1}}



% Set up the appearance of todo notes
\newcommand{\todonote}[1]{\todo[backgroundcolor=white, linecolor=black, 
                                size=\small, noline]{#1}}



% Environment for showing bits of code, terminal listings, and so on.
\definecolor{box-grey}{gray}{0.4}
\newcommand{\codelabel}[1]{\textcolor{black}{#1}}
\DefineVerbatimEnvironment
    {Code}
    {Verbatim}
    {frame=single, fontsize=\scriptsize, framesep=3mm, formatcom=\vspace{7pt}, 
      xleftmargin=5mm, xrightmargin=5mm, rulecolor=\color{box-grey},
      labelposition=topline, numbersep=8pt, baselinestretch=1.2,
      numbers=left, samepage=true}
% N.B. can use option label=\codelabel{??} as needed.



% Wrapper for the above for teminal listings
% This is a pretty nasty hack, but I'm really tired. ;)
\definecolor{shadecolor}{gray}{0.95}
\setlength{\FrameSep}{-5mm}
\newenvironment{Term}
    {\begin{center}\begin{Sbox}\begin{minipage}{\linewidth-17.4mm}}
    {\end{minipage}\end{Sbox}{\setlength{\fboxsep}{3mm}\vspace*{8pt}\fcolorbox{box-grey}{shadecolor}{\TheSbox}\vspace*{6pt}}\end{center}}

\DefineVerbatimEnvironment
    {CodeTerm}
    {Verbatim}
    {frame=none, fontsize=\scriptsize, framesep=3mm, formatcom={}, 
      xleftmargin=0mm, xrightmargin=0mm, rulecolor=\color{box-grey},
      labelposition=topline, numbersep=8pt, baselinestretch=1.2,
      numbers=left, samepage=true,
      numbers=none}

% Usage:
%\begin{Code}
%\begin{CodeTerm}
%Hello, World.
%\end{CodeTerm}
%\end{Term}


% Rename the abstract
\renewcommand{\abstractname}{Introduction}



% Remove section numbering
\setcounter{secnumdepth}{0}



% Remove bold and add dots to sections in the TOC
\renewcommand\cftsecfont{\normalfont}
\renewcommand\cftsecpagefont{\normalfont}
\renewcommand{\cftsecleader}{\cftdotfill{\cftsecdotsep}}
\renewcommand\cftsecdotsep{\cftdot}
\renewcommand\cftsubsecdotsep{\cftdot}
% Show only sections in the TOC
\setcounter{tocdepth}{1}


% Configure the \maketitle command a little
\setlength{\droptitle}{-30pt}
\predate{\begin{center}\small}
\postdate{\par\end{center}}



% Set up the desired date format (20^th July 2011)
% N.B. this is very similar to the default with the 'nodayofweek' option.
\newdateformat{mydate}{\ordinaldate{\THEDAY} \monthname[\THEMONTH] \THEYEAR}
\mydate



% New float type for code listings
\floatstyle{plain}
\floatname{listing}{Listing}
\newfloat{listing}{p}{lol}



% Set up appearance of captions
\captionsetup{margin=30pt,font=small,labelfont=bf}



% Remove ugly boxes etc around hyperlinks.
\hypersetup{pdfborder = {0 0 0}}




% Tree Drawing
% {A, B, {C, D}, {E, F}}
\newcommand{\treeDrawABCDEF}{%
    \[\begin{tikzpicture}
        \node{$\{A, B\}$}
            child{node {$\{C, D\}$}}
            child{node {$\{E, F\}$}}
        ;
    \end{tikzpicture}\]
}




% Add a command to produce nice looking tildes: \urltilde
% http://stackoverflow.com/questions/718479/what-is-the-best-way-to-produce-a-tilde-in-latex-for-a-website/3900556#3900556
\catcode`~=11 % make LaTeX treat tilde (~) like a normal character
\newcommand{\urltilde}{\kern -.15em\lower .7ex\hbox{~}\kern .04em}
\catcode`~=13 % revert back to treating tilde (~) as an active character




% Allow document title to be set from main file.
\newcommand{\thedocumenttitle}{}
\newcommand{\documenttitle}[1]{%
    \renewcommand{\thedocumenttitle}{#1}
}



% Title Info
\title{Flamingo Auto-Tuning\\\decofourleft~~~\thedocumenttitle~~~\decofourright}
\author{Ben Spencer\\\normalsize\href{mailto:ben@mistymountain.co.uk}{ben@mistymountain.co.uk}}
\date{Last updated: \today}




\documenttitle{Developer's Guide}



\begin{document}

\maketitle

\begin{abstract}
This guide provides an introduction to how the auto-tuner was designed and 
developed. It should be helpful to anyone who wants to understand how the 
tuner works, whether to use it more effectively, or modify it to better suit 
your needs.

If you have any comments or questions about the tuner or this guide then
please feel free to get in touch.
\end{abstract}

\tableofcontents



\section{Goals}
The tuner is implemented in Python, and requires at least version 2.5.

Flexibility has been the overriding goal from the start. The programmer can 
specify arbitrary shell commands which are used to compile, run and score 
tests. This guide will give an overview of the high-level design of the tuner: 
how it is split into modules and what each is designed to do, as well as how 
they interact and together create the overall system. 

I have tried to keep separate parts of the system independent of each other, 
which has mostly succeeded, but some different parts do rely heavily on the 
behaviour of others, even when they have been abstracted into separate modules.




\clearpage

\section{Original Project Report}
I originally began work on the tuner as a third year project towards my 
undergraduate degree. My project 
report\footnote{\url{http://people.maths.ox.ac.uk/~gilesm/op2/autotuning-2011-05-30.pdf}} 
(which is very long) contains some very detailed information on the design and 
development of the system, especially the optimisation algorithm, which I 
explain in detail and prove correct. 

It is worth remembering that the report was based 
on an older version of the tuner (0.11), so some parts are very different 
(notably tests are now run by the \class{Evaluator} class, rather than 
`evaluation functions') but the broad architecture of the system and many of 
the details (for example the optimisation algorithm) remain the same.





\section{High-Level Description}
This section details the main classes providing the tuner's core 
functionality:

\begin{description}
    
    \item[\class{VarTree}] \hfill\\
        This class represents the variable trees which list the variables to 
        be tuned and which are independent of each other. This structure is 
        described (from a user's perspective) in more detail in the 
        User's Guide (\filename{doc/user.pdf}). 
        
        The variables are supplied to the system in a nested braces format:
        \[\{A,~B,~\{C,~D\},~\{E,~F\}\}\]
        This says that the variables \var{A} and \var{B} are dependent on each 
        other, and on \var{C}, \var{D}, \var{E} and \var{F}, but that the sets 
        \var{\{C,~D\}} and \var{\{E,~F\}} are independent. Being independent 
        means that these sets can be tuned separately: if an optimal valuation 
        of \var{\{C,~D\}} is found at one setting of \var{\{E,~F\}}, then this 
        will still be optimal for any other valuation of \var{\{E,~F\}}.
        
        This notation represents a tree of variables, where each node contains 
        a set of dependent variables and a set of subtrees which are all 
        mutually independent, but which depend on the `parent' variables. The 
        above example would be represented:
        \treeDrawABCDEF
        
        The \class{VarTree} class represents these trees. Each instance 
        represents a tree node and they are linked together to represent the 
        entire tree. The objects themselves do not require much extra 
        information or functionality, they only contain a list of top-level 
        variables and a list of subtrees. The only methods provided are to 
        flatten the tree into a list of variables. All other operations on 
        \class{VarTree} objects are provided as outside functions which 
        manipulate them (the most interesting being \codefragment{treeprint()}, 
        which returns a string to display the tree structure in a terminal).
        
        
        
    \item[\class{Optimisation}] \hfill\\
        This class defines the optimisation algorithm used by the tuner. 
        This algorithm exploits variable independence (given by a 
        \class{VarTree}) to reduce the number of tests required, while still 
        being comprehensive. It is parameterise by a \class{VarTree} Instance, 
        listing the variables to tune and their independence; a list of the 
        possible values of each variable; and an \class{Evaluator} instance, 
        which handles the actual running of the tests.
        
        The algorithm used is recursive, following the structure of the 
        \class{VarTree}. At leaf nodes, every possible combination of the 
        variables is tested by brute force. At branch nodes, the system 
        recognises that each subtree is independent, so optimises them 
        separately. For each possible valuation of the top-level variables, 
        it recursively optimises the subtrees, one-by-one. The score when each 
        subtree is optimised gves the best possible score for this setting of 
        top-level variables. Once all the top-level valuations have been tried, 
        the one which gave the best score is chosen and returned, along with 
        it's subtree-optimums.
        
        
        
    \item[\class{Evaluator}] \hfill\\
        This class is used by \class{Optimisation} to actually test different 
        settings of variables, and to keep a log of the tests, so the results 
        can be looked up as they are needed. The \class{Evaluator} is 
        parameterised by the commands required to compile and test a 
        particular test, and how to calculate the overall score if repeat 
        tests are being run. This class performs all the direct interaction 
        with the shall and the tests which are run (e.g. reading their output 
        and checking their return code).
        
        When the optimisation algorithm reaches a leaf node of the 
        \class{VarTree}, it submits a whole group of tests to the 
        \class{Evaluator} at once. At the moment, these tests are simply run 
        in sequence, compiling one, testing it a number of times and then 
        cleaning it; before moving on to the next. However, because thests are 
        submitted in groups, there is scope for other implementations of 
        \class{Evaluator} to provide different evaluation strategies, such as 
        running tests interleaved with other tests, to reduce the effects of 
        inconsistent load on the system.
        
        The log of tests performed is kept for two reasons. Firstly, it is 
        used by the optimiser to check the scores of tests which were 
        submitted. Secondly, it is used at the end of testing to create the 
        \filename{.csv} log file listing all tests performed.
        
    
\end{description}



\clearpage

\section{A Tuning Run}
This section describes a run of the progam from start to finish, showing how 
the different parts of the system are connected and work together to provide 
the final result.

The main program is \filename{tune.py}. When the user runs this, they provide 
a configuration file as an argument. The first step is that 
\filename{tune\_conf.py} is used to read this configuration file (using the 
Python module \codefragment{ConfigParser}) and validate the settings.

Next, an instance of \class{Evaluator} is created, parameterised by the 
compilation and testing commands, and the number of test repetitions.

An instance of \class{Optimisation} is created, and provided with the 
\class{VarList} and list of possible values read from the configuration file, 
and the \class{Evaluator} which it can use to run tests. 

The system prints out a description of the information read from the 
configuration file and the tuning it is about to perform, 
then begins the tuning.

The optimiser works recursively over the structure of the \class{VarTree} and 
at each leaf node it submits a group of tests to the \class{Evaluator}. The 
\class{Evaluator} runs each test in sequence, logging the results. If each 
test is to be repeated, it runs them and calculates the overall score.

If the tests are to be timed by the system, the \class{Evaluator} does this 
timing to determine a test's score. If the tests use a custom figure-of-merit, 
then the \class{Evaluator} captures their output and checks the final line 
for the score, which is interpreted as a float.

The scores for each test are checked by the optimiser to find the best setting 
of the variables at that leaf node, which it selects and continues optimising 
the rest of the tree. As higher level variables are changed, it will be 
necessary to return to the leaves and fin new optimal settings for the changes 
in higher-level variables.

Once the optimisation is complete, the optimiser returns the best valuation 
found, as well as the score for that valuation, and the number of tests 
performed.

The log of testing is passed from the \class{Evaluator} to 
\filename{logging.py} and written out as a \filename{.csv} file, which 
contains all the details of the testing.

The \filename{.csv} log file can then be processed (by the user, not 
automatically) with any of the log analysis utilities. These can be used to 
generate graphs of the testing process, to show (for example) which variables 
had the greatest effect.






\clearpage

\section{File Listing}


\begin{description}


    \item[\filename{autotune}] \hfill\\
        The tuner (A link to \filename{tuner/tune.py}).
        
    \item[\filename{doc/}] \hfill\\
        Documentation for the tuner.
        
        \begin{description}
            \item[\filename{dev.pdf}] \hfill\\
                Developer's documentation (this document).
                
            \item[\filename{tutorial.pdf}] \hfill\\
                A Beginner's Tutorial. Leads them through the setup and tuning 
                of the matrix-multiplication test case.
                
            \item[\filename{user.pdf}] \hfill\\
                User's guide. A more comprehensive reference detailing all the 
                features and abilities of the tuner.
                
                
        \end{description}
        
        
    \item[\filename{examples/}] \hfill\\
        Example programs to demonstrate the system's use. Each comes with a 
        sample configuration file which can be used to tune it, and most 
        have some sample results from this tuning.
        
        \begin{description}
            \item[\filename{hello}] \hfill\\
                A simple test case which compiles a `hello world' program 
                with different levels of compiler optimisation.
                
            \item[\filename{laplace3d}] \hfill\\
                A CUDA test case. Compiles and tests a version of the 
                \filename{laplace3d} CUDA example from Mike Giles' CUDA 
                programming course\footnote{\url{http://people.maths.ox.ac.uk/gilesm/cuda/}}.
                
            \item[\filename{looping}] \hfill\\
                A simple test case where parameters control the number of loop 
                iterations performed in the program.
                
            \item[\filename{maths}] \hfill\\
                A simple test case where parameters are summed using the 
                \command{expr} command. Demonstrates the use of a custom 
                figure-of-merit.
                
            \item[\filename{matlab}] \hfill\\
                A MATLAB program being tuned to determine the optimum 
                level of `strip-mining' vectorisation.
                
            \item[\filename{matrix}] \hfill\\
                A blocked matrix-matrix multiplication test case, which 
                is tuned to find the optimal block size.
                
        \end{description}
        
        
    \item[\filename{README\_Dev}] \hfill\\
        Breif intro for developers, contains a file listing and change log.
        
    \item[\filename{README\_User}] \hfill\\
        Breif intro for users, mostly points to the proper documentation.
        
    \item[\filename{tuner/}] \hfill\\
        The Python source code for the tuner itself.
        
        \begin{description}
            \item[\filename{evaluator.py}] \hfill\\
                Defines the \class{Evaluator} class, which controls how tests 
                are evaluated, handling all compilation, execution and timing 
                of tests. It also keeps a log of all tests performed, 
                for use by the optimiser.

            \item[\filename{helpers.py}] \hfill\\
                Provides several small helper functions, 
                which are used elsewhere.

            \item[\filename{logging.py}] \hfill\\
                Provides a function to output the testing log as a 
                \filename{.csv} file.

            \item[\filename{optimisation\_bf.py}] \hfill\\
                A brute force optimiser, \class{OptimisationBF}, implementing 
                the same methods as \class{Optimisation}. THis is used for 
                testing, and in the system's self-demonstration.

            \item[\filename{optimisation.py}] \hfill\\
                The optimisation algorithm. Defines the \class{Optimisation} 
                class. This class sets up and runs the recursive optimisation 
                algorithm, which exploits variable independence to reduce 
                the number of tests that must be performed.

            \item[\filename{output.py}] \hfill\\
                Controls where the tuner's output is sent. This can be printed 
                to the screen or saved to a file. Sets up the three output 
                possiblities \codefragment{all}, \codefragment{short} and 
                \codefragment{full} for the tuner to use.

            \item[\filename{test\_evaluations.py}] \hfill\\
                Provides a `dummy' \class{Evaluator} class, 
                \class{FuncEvaluator}. This implements the same interface 
                as \class{Evaluator}, but uses an `artifical' evaluation 
                function to score each test, rather than running any shell 
                commands. This is used for the system's self-demonstration, 
                as well as for testing. The evaluation function respects the 
                variable independence of the problem being solved.

            \item[\filename{testing.py}] \hfill\\
                Checks \class{Optimisation} against \class{OptimisationBF} for 
                a couple of different inputs. This is used as a demonstration 
                of the system.

            \item[\filename{tune\_conf.py}] \hfill\\
                Reads settings from the config file and performs 
                some validation.

            \item[\filename{tune.py}] \hfill\\
                The main script. Reads the config file from the command line, 
                sets up and runs the optimisation and reports the final 
                reusults to the user.

            \item[\filename{vartree\_parser.py}] \hfill\\
                Defines a \class{VarTree} parser, which was generated by  
                \command{wisent}, a Python 
                parser-generator\footnote{\url{http://seehuhn.de/pages/wisent}}. 
                This is used by \filename{vartree.py} to 
                convert token strings into parse trees. The language grammar 
                was constructed so that these parse trees have the same 
                structure as the corresponding \class{VarTree}, making the 
                conversion simple.

            \item[\filename{vartree.py}] \hfill\\
                Defines the \class{VarTree} class and several related 
                functions. These include a parser for creating \class{VarTree} 
                objects from strings and a function to print a \class{VarTree} 
                in a tree representation.

        \end{description}
        
        
    \item[\filename{utilities/}] \hfill\\
        Contains utilities to analyse nd visualise the tuning results.
        
        \begin{description}
                \item[\filename{common.py}] \hfill\\
                    Some helper functions for reading in the CSV file.
                    
                \item[\filename{csv\_plot.m}] \hfill\\
                    A matlab script to draw plots.
                    
                \item[\filename{output\_gnuplot.py}] \hfill\\
                    Outputs a \command{gnuplot} script for plotting graphs.
                    
                \item[\filename{output\_screen.py}] \hfill\\
                    Displays a graph on screen with \filename{matplotlib} 
                    (a Python graph plotting library).
                    
        \end{description}
        
        
\end{description}







\end{document}
